\begin{frame}{Computação de conjuntos $FOLLOW$}

{\bf Terminal}\\
Se $a$ é um terminal, então $FOLLOW(a) = \{a\}$.\bigskip

{\bf Vazio}\\
$FOLLOW(\epsilon) = \{\epsilon\}$.\bigskip

{\bf Não-terminal (Variável)}\\
\begin{enumerate}
 \item Se $X$ é uma variável e $X \rightarrow Y_1 Y_2 ... Y_k$ é uma produção, 
      então tudo que está em $FOLLOW(Y_1)$ está em $FIRST(X)$, exceto $\epsilon$. 
 \item Se $Y_1$ pode derivar $\epsilon$, então tudo que está em $FOLLOW(Y_2)$ 
       também está em $FOLLOW(X)$, exceto $\epsilon$. 
\item Continuar esse processo até que um $Y_i$ não possa 
      derivar $\epsilon$ ou até que todos os $Y_i$ tenham sido processados. 
\item Se todos os $Y_i$ podem derivar $\epsilon$, então adicionar $\epsilon$ a $FOLLOW(X)$.
\end{enumerate}

\end{frame}

\begin{frame}{Computação de conjuntos $FOLLOW$}{Exemplo}
    \begin{columns}
        \begin{column}{0.3\textwidth}
            \begin{enumerate}
              \item $E \rightarrow T S$

              \item $S \rightarrow + T S\ |\ \epsilon$

              \item $T \rightarrow F U\ |\ \epsilon$

              \item $U \rightarrow * F U\ |\ \epsilon$

              \item $F \rightarrow ( E )\ |\ int$
            \end{enumerate}
        \end{column}    
        \pause
        \begin{column}{0.7\textwidth}
            \begin{itemize}
                \item $FOLLOW(F) = \{(, int, \epsilon\}$
                \item $FOLLOW(U) = \{*, \epsilon\}$
                \item $FOLLOW(T) = FIRST(F) \cup {\epsilon} = \{(, int, \epsilon\}$ 
                \item $FOLLOW(S) = \{+, \epsilon\}$
                \item $FOLLOW(E) =  FIRST(T) = \{(, int, \underline{\epsilon}\} \cup FIRST(S) = \{(, int, +, \epsilon\}$
            \end{itemize}
        \end{column}    
    \end{columns}
\end{frame}

\begin{frame}{Computação de conjuntos $FOLLOW$}

    $FOLLOW(A)$ é o conjunto de terminais que podem aparecer 
depois do não-teminal $A$, incluindo o símbolo de fim de entrada
$\$$, se $A$ pode ser o último símbolo em alguma derivação.\bigskip

{\bf Regras para computar $FOLLOW$}:\medskip

 \begin{enumerate}
    \item Se $S$ é o símbolo inicial então $FOLLOW(S)=\{\$\}$.
    \item Se $A \rightarrow \alpha B\beta$ então adicionar FOLLOW($\beta$) (exceto $\{\epsilon\}$) a $FOLLOW(B)$.
    \item Se $A \rightarrow \alpha B$ ou $A \rightarrow \alpha B\beta$ onde $\epsilon \in FOLLOW(\beta)$, 
            então adicionar $FOLLOW(A)$ a $FOLLOW(B)$.
 \end{enumerate}

\end{frame}

\begin{frame}{Computação de conjuntos $FOLLOW$}{Exemplo}
    \begin{columns}
        \begin{column}{0.3\textwidth}
            \begin{enumerate}
              \item $E \rightarrow T S$

              \item $S \rightarrow + T S\ |\ \epsilon$

              \item $T \rightarrow F U\ |\ \epsilon$

              \item $U \rightarrow * F U\ |\ \epsilon$

              \item $F \rightarrow ( E )\ |\ int$
            \end{enumerate}
        \end{column}    
        \pause
        \begin{column}{0.7\textwidth}
            \begin{itemize}
                \item $FOLLOW(E) = \{\$, )\}$
                \item $FOLLOW(S) = \{\$, )\}$
                \item $FOLLOW(T) = \{+, \$, )\}$
                \item $FOLLOW(U) = \{+, \$, )\}$
                \item $FOLLOW(F) = \{*, + \$, )\}$
            \end{itemize}
        \end{column}    
    \end{columns}
\end{frame}

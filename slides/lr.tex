\begin{frame}{Computação de conjuntos $FIRST$}

{\bf Terminal}\\
Se $a$ é um terminal, então $FIRST(a) = \{a\}$.\bigskip

{\bf Vazio}\\
$FIRST(\epsilon) = \{\epsilon\}$.\bigskip

{\bf Não-terminal (Variável)}\\
\begin{enumerate}
 \item Se $X$ é uma variável e $X \rightarrow Y_1 Y_2 ... Y_k$ é uma produção, 
      então tudo que está em $FIRST(Y_1)$ está em $FIRST(X)$, exceto $\epsilon$. 
 \item Se $Y_1$ pode derivar $\epsilon$, então tudo que está em $FIRST(Y_2)$ 
       também está em $FIRST(X)$, exceto $\epsilon$. 
\item Continuar esse processo até que um $Y_i$ não possa 
      derivar $\epsilon$ ou até que todos os $Y_i$ tenham sido processados. 
\item Se todos os $Y_i$ podem derivar $\epsilon$, então adicionar $\epsilon$ a $FIRST(X)$.
\end{enumerate}

\end{frame}

\begin{frame}{Computação de conjuntos $FIRST$}{Exemplo}
    \begin{columns}
        \begin{column}{0.3\textwidth}
            \begin{enumerate}
              \item $E \rightarrow T S$

              \item $S \rightarrow + T S\ |\ \epsilon$

              \item $T \rightarrow F U\ |\ \epsilon$

              \item $U \rightarrow * F U\ |\ \epsilon$

              \item $F \rightarrow ( E )\ |\ int$
            \end{enumerate}
        \end{column}    
        \pause
        \begin{column}{0.7\textwidth}
            \begin{itemize}
                \item $FIRST(F) = \{(, int\}$
                \item $FIRST(U) = \{*, \epsilon\}$
                \item $FIRST(T) = FIRST(F) \cup \{\epsilon\} = \{(, int, \epsilon\}$
                \item $FIRST(S) = \{+, \epsilon\}$
                \item $FIRST(E) =  FIRST(T) = \{(, int, \underline{\epsilon}\} \cup FIRST(S) = \{(, int, +, \epsilon\}$
            \end{itemize}
        \end{column}    
    \end{columns}
\end{frame}

\begin{frame}{Computação de conjuntos $FOLLOW$}

    $FOLLOW(A)$ é o conjunto de terminais que podem aparecer 
depois do não-teminal $A$, incluindo o símbolo de fim de entrada
$\$$, se $A$ pode ser o último símbolo em alguma derivação.\bigskip

{\bf Regras para computar $FOLLOW$}:\medskip

 \begin{enumerate}
    \item Se $S$ é o símbolo inicial então $FOLLOW(S)=\{\$\}$.
    \item Se $A \rightarrow \alpha B\beta$ então adicionar FIRST($\beta$) (exceto $\{\epsilon\}$) a $FOLLOW(B)$.
    \item Se $A \rightarrow \alpha B$ ou $A \rightarrow \alpha B\beta$ onde $\epsilon \in FOLLOW(\beta)$, 
            então adicionar $FOLLOW(A)$ a $FOLLOW(B)$.
 \end{enumerate}

\end{frame}

\begin{frame}{Computação de conjuntos $FOLLOW$}{Exemplo}
    \begin{columns}
        \begin{column}{0.3\textwidth}
            \begin{enumerate}
              \item $E \rightarrow T S$

              \item $S \rightarrow + T S\ |\ \epsilon$

              \item $T \rightarrow F U\ |\ \epsilon$

              \item $U \rightarrow * F U\ |\ \epsilon$

              \item $F \rightarrow ( E )\ |\ int$
            \end{enumerate}
        \end{column}    
        \pause
        \begin{column}{0.7\textwidth}
            \begin{itemize}
                \item $FOLLOW(E) = \{\$, )\}$
                \item $FOLLOW(S) = \{\$, )\}$
		\item $FOLLOW(T) = \{+, ), \$ \}$
		\item $FOLLOW(U) = \{+, ), \$ \}$
                \item $FOLLOW(F) = \{*, +, ), \$\}$
            \end{itemize}
        \end{column}    
    \end{columns}
\end{frame}

\begin{frame}{Analisador LR(0)}{Construção do autômato}


 {\bf Fecho ({\it closure\/})} dos conjuntos de itens\\\bigskip

Se $I$ é um conjunto de itens da gramática $G$, então o
$CLOSURE(I)$ é o conjunto de itens construído de $I$
pela aplicação de 2 regras:

	\begin{enumerate}
		\item Inicialmente, adicionar cada item de $I$ ao $CLOSURE(I)$.
		\item Se $A\rightarrow \alpha\ .\ B\beta$ está iem $CLOSURE(I)$ e
		$B\rightarrow \gamma$ ao $CLOSURE(I)$ é uma regra da gramática, 
		então adicionar o item $B\rightarrow\ .\ \gamma$ a $CLOSURE(I)$.
		Aplicar essa regra até que não haja novos itens a serem
		adicionados ao $CLOSURE(I)$.
	\end{enumerate}

\end{frame}

% Thain
\begin{frame}[fragile]{Analisador SLR}{{\it Parser simple} LR (SLR)}

{\bf Tabela para a criação do analisador SLR}\bigskip

	Dada uma gramática $G$ e um autômato correspondente LR(0),
	criar as tabelas de $ACTION(s,a)$ e $GOTO(s,A)$ para todos
	os estados $s$, terminais $a$ e não-terminais em $G$.\bigskip

\lstset{mathescape=true
	,morekeywords={goto,reduce,shift}
	,basicstyle=\small}
\begin{lstlisting}
Para cada estado s:
   Para cada item $A \rightarrow \alpha\ .\ B\beta$:
     $ACTION(s,a) = $ desloca shift para o estado $t$ de acordo com  LR(0)
   Para cada item $A\rightarrow \alpha\ .\ B\beta$:
     $GOTO(s,B) = $ vai (goto) para o estado $t$ de acordo com  LR(0)
   Para cada item $A \rightarrow \alpha$:
     Para cada termina $a \in FOLLOW(A)$:
       $ACTION(s,a) = $ reduz (reduce) pela regra $A \rightarrow \alpha$

Todos os estados remanescentes são considerados erros de estados.
\end{lstlisting}

\end{frame}

\begin{frame}[fragile]{Algoritmo do analisador sintático ({\it parser\/}) SLR}

\lstset{mathescape=true
	,morekeywords={goto,reduce,shift}}
\begin{lstlisting}
pilha $S$
a $\leftarrow$ primeiro token da entrada

Loop:
  s $\rightarrow$ topo da pilha
  Se $ACTION[s,a] =$ primeira produção:
    Estado de aceitação
  Senão se $ACTION[s,a] =$ shift $t$:
    Empilhar estado $t$
    $a \leftarrow$ próximo token da entrada
  Senão se $ACTION[s,a] =$ reduce $A\rightarrow \beta$:
    Desempilhar os estados correspondentes a $\beta$
    $t \leftarrow$ topo da pilha
    Empilhar $GOTO[t,A]$
  Caso contrário:
    Abortar a análise sintática
\end{lstlisting}

\end{frame}


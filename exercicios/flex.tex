\section*{Análise léxica}

\paragraph{1.} Descreva o comportamento das expressões regulares a
seguir e dê alguns exemplos de {\it strings\/} que se encaixam em cada
padrão.

\begin{enumerate}
\item {\tt [+-]*[0-9]+$\backslash$.[0-9]+}

\item {\tt [1-9][0-9]*|0}

\item {\tt [A-Z]+}

\item {\tt [a-zA-Z\_]+}

\item {\tt [a-zA-Z][a-zA-Z0-9\_]*}

\item {\tt [az-]+}

\item {\tt "[a-z]"}

\item {\tt [0-9]+[}$\backslash${\tt t]*}$\backslash${\tt *[}$\backslash${\tt t]*[0-9]+}

\item {\tt x(cachorro|gato)x}

\item {\tt cachorro.*gato}

\item {\tt [gato]}

\item {\tt [gato]+}

\item {\tt (ab|cd)?(ef)*}

\item {\tt (a|b)*a(a|b)}

\item {\tt (a|b)*a(a|b)(a|b)}

\item {\tt (a|b)*a(a|b)(a|b)(a|b)}

\item {\tt (abcd|abc)/d}

\item {\tt (a|ab)/ba}

\item {\tt aa*|a*}

\item {\tt [$\backslash$\^{}$\backslash$+$\backslash$-$\backslash$:$\backslash$*$\backslash$]]}
\end{enumerate}

\paragraph{2.} Implemente os seguintes programas usando o analisador léxico {\tt flex}.

\begin{enumerate}[a)]
\item Imprima o número de palavras que comecem com letras maiúsculas
  de um texto fornecido pela entrada-padrão ou arquivo (daqui pra
  frente só texto fornecido).

\item Imprima o número de letras maiúsculas, minúsculas e números de
  um texto fornecido.

\item Imprima o número de vogais e consoantes de um texto fornecido.

\item Que tenha comportamento parecido com o do programa
  \href{https://pt.wikipedia.org/wiki/Wc}{\tt wc} que imprime o número
        de caracteres (bytes), palavras (somente letras) 
        e número de linhas de um texto fornecido.

\item Encontre os números inteiros e de ponto flutuante a partir de um 
texto fornecido. Implemente uma função chamada {\tt install\_num}, 
que converta a {\it string\/}, que casa com as expressões regulares para números, 
para número inteiro ou ponto flutuante e imprima-o na saída-padrão.

\item A partir de um texto fornecido, imprima os valores existentes no
  texto em moeda brasileira, por exemplo, {\tt R\$ 12,50}, {\tt
    R\$312,78} ou {\tt R\$ 0,62}.  Se a entrada fornecida for

\begin{verbatim}
A calça original custa R$ 78,50, porém o terno custa em
torno de R$700,00 a   R$  1250,00. A camisa social sai
em torno de R$ 73,25. Se o pagamento for em cartão há
uma taxa adicional de R$ 0,80 a R$ 5,00.
\end{verbatim}

o programa deverá imprimir

\begin{verbatim}
R$ 78,50
R$700,00
R$  1250,00
R$ 73,25
R$ 0,80
R$ 5,00
\end{verbatim}

Dica: utilize um {\it buffer\/} (vetor de {\it strings\/}) para
armazenar os textos contendo os valores.

\item Receba um arquivo contendo código fonte em C e imprima a quantidade de:
    - palavras reservadas: {\tt if}, {\tt else}, {\tt while}, {\tt do}, {\tt switch} e {\tt case};
    - funções;
    - números inteiros;
    - caracteres especiais: "{\tt !}", "{\tt @}", "{\tt *}", "{\tt \&}", "{\tt |}", "{\tt \%}", "{\tt \$}" e "{\tt \#}".

Após a compilação do programa e supondo que o binário gerado
chama-se {\tt a.exe} (Windows) ou {\tt a.out} (Linux),
forneça um arquivo com código C como argumento
da seguinte forma no terminal (PowerShell, bash, ...):

\begin{verbatim}
# Windows
.\a.exe arquivo.c
# Linux
./a.out arquivo.c
\end{verbatim}

\item (Adaptado de Aho, 2007) Que analise uma expressão SQL e
  reconheça as palavras-chave {\tt SELECT}, {\tt FROM} e {\tt WHERE}
  em qualquer combinação de maiúsculas ou minúsculas, bem como os
  identificadores existentes na expressão.  Não há necessidade de
  adicionar o identificador em um tabela de símbolos, mas descreva
  como seria o tratamento dos identificadores levando em conta que a
  diferença entre maiúsculas e minúsculas é ignorada.
\end{enumerate}

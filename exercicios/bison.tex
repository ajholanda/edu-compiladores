\section*{Análise sintática com bison (yacc)}

\newexerc~Para a gramática da calculadora de mesa simples

\begin{tabbing}
  E \= $\rightarrow$ \= E + T \= | T \\
  T \> $\rightarrow$ \> T * F \> | F\\
  F \> $\rightarrow$ \> ( E ) \> | {\bf digit}\\
\end{tabbing}

\noindent Faça a tabela de reduções e deslocamentos, listando os
valores da entrada e pilha a cada ação usando o {\it parser\/} LALR
para as seguintes expressões:

\begin{enumerate}[a.]
\item $1 + 2$
\item $1 * 2 + 5$
\item $1 * 2 + 5 * 3$
\item $1 * 2 + 5 + 3$
\item $1 * (2 + 5) + 3$
\item $1 * 2 + 5 + 3 * 4$
\item $1 * 2 + 5 * (3 + 4)$
\end{enumerate}

\newexerc~Escreva um programa de ``calculadora de mesa'' {\tt yacc} que avalie
expressões aritméticas usando a \href{https://pt.wikipedia.org/wiki/Nota%C3%A7%C3%A3o_polonesa_inversa}{notação polonesa inversa}.
  [\href{https://github.com/aholanda/edu-compiladores/tree/main/bison/03-calc}{Solução}]

\newexerc~Escreva um programa de ``calculadora de mesa'' {\tt yacc}
que avalie expressões aritméticas usando a notação da linguagem
\href{https://pt.wikipedia.org/wiki/Lisp}{LISP}, com a lista limitada
a dois operandos. Por exemplo, a expressão

\begin{center}
\begin{verbatim}
(/ (* (- 3 1) 4) 2)
\end{verbatim}
\end{center}

\noindent tem como sequência de avaliação

\begin{center}
\begin{verbatim}
(/ (* 2 4) 2) =
(/ 8 2) =
4
\end{verbatim}
\end{center}

\noindent   [\href{https://github.com/aholanda/edu-compiladores/tree/main/bison/04-calc}{Solução}]

\newexerc~[Aho] Escreva um programa de "calculadora de mesa" {\tt
  yacc} que avalie expressões Booleanas tendo as operações
representadas pelos caracteres

\begin{itemize}
\item {\tt '|'} $\rightarrow$ {\tt OR},
\item {\tt '\&'} $\rightarrow$ {\tt AND},
\item {\tt '\^{ }'} $\rightarrow$ {\tt XOR},
\item {\tt '\textasciitilde'} $\rightarrow$ {\tt NOT}.
\end{itemize}

\noindent O operador unário $\sim$ possui maior
precedência sobre os demais operadores, sendo que estes possuem mesma
precedência excetuando se a expressão estiver entre parênteses. Por
exemplo, a expressão

\begin{verbatim}
0 ^ (1 | 0) | ~0
\end{verbatim}

\noindent tem como sequência de avaliação

\begin{verbatim}
0 ^ (1 | 0) | 1 =
0 ^ 1 | 1 =
1 | 1 =
1
\end{verbatim}

\noindent [\href{https://github.com/aholanda/edu-compiladores/tree/main/bison/05-bool}{Solução}]

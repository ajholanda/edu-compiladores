\documentclass{article}

\usepackage{xcolor}
\definecolor{codegreen}{rgb}{0,0.6,0}
\definecolor{codegray}{rgb}{0.5,0.5,0.5}
\definecolor{codepurple}{rgb}{0.58,0,0.82}
\definecolor{backcolour}{rgb}{0.95,0.95,0.92}
\usepackage{polyglossia}
\setdefaultlanguage{brazil}
\usepackage{listings}
\renewcommand{\lstlistingname}{Listagem}
\lstdefinestyle{xyzstyle}{
    morekeywords={fn,return,var,while},
    backgroundcolor=\color{backcolour},   
    commentstyle=\color{codegreen},
    keywordstyle=\color{blue},
    numberstyle=\tiny\color{codegray},
    stringstyle=\color{codepurple},
    basicstyle=\ttfamily\footnotesize,
    breakatwhitespace=false,         
    breaklines=true,                 
    captionpos=b,                    
    keepspaces=true,                 
    numbers=left,                    
    numbersep=5pt,                  
    showspaces=false,                
    showstringspaces=false,
    showtabs=false,                  
    tabsize=2
}

\lstset{style=xyzstyle}

\date{\today}
\def\lang{{\sc XYZ}}

\title{Compiladores: Projeto}
\author{Zhao Liang}
\date{2022-10-18}
\begin{document}
\maketitle

\section{A linguagem \lang}

\subsection{Descrição}

\begin{itemize}
\item Uma linguagem imperativa com a sintaxe parecida com a 
linguagem C;
\item Um programa \lang{} é uma sequência de definições de 
funções sem recursividade e sem variáveis globais;
\item A função principal por onde é iniciada a execução 
das instruções chama-se {\tt main} que retorna um valor e 
não possui parâmetros;
\item Só ha um tipo de dados na linguagem, o tipo {\bf inteiro}, 
onde na avalização Booleana o {\tt 0} é falso e qualquer número 
diferente de {\tt 0} corresponde ao Booleano verdadeiro.
\end{itemize}

\paragraph{Restrições.} A linguagem é bem restrita, não havendo 
vetores, ponteiros nem módulos. Todo o programa é definido em um 
único arquivo.

\subsection{Funções}

\paragraph{Sintaxe}
\begin{itemize}
    \item Uma definição de função começa com a palavra reservada 
    {\bf fn} seguida pelo seu {\bf nome}, uma lista de parâmetros 
    entre parênteses e um corpo.
    \item Todas as funções devem retornar um valor mesmo que este 
    valor não seja usado diretamente. O retorno do valor é realizado 
    usando o comando {\bf return}.
    \item  Espaço em branco ` ', tabulação `$\backslash$t', 
	    retorno de ``carro'' `$\backslash$r' e nova linha
		`$\backslash$n' devem ser ignorados.
\end{itemize}


\subsection*{Proposição ({\it statement})}

Uma proposição na linguagem \lang{} pode ser:

\begin{enumerate}
\item proposição vazia;
\item declaração de variáveis ({\it declarations});
\item atribuição ({\it assignment});
\item incremento e decremento;
\item retorno de função;
\item chamada de função ({\it function call});
\item desvio de fluxo ({\tt if} com e sem {\tt else})
\item laço ({\it loop}, só há um, o {\tt while}).
\end{enumerate}

\paragraph{Identificadores ({\it identifiers}).}
Um identificador (nome) é uma letra, opcionalmente seguida
por letras e {\it underscores} ``\_''.

\paragraph{Declarações.}
Uma variável deve ser declarada antes de ser usada começando
pela palavra reservada {\bf var} e uma lista de atribuições 
separada por ponto e vírgula ``;''. Nenhuma variável pode ser 
declarada sem possuir um valor inicial.

\subsection*{Expressões}
As seguintes espressões exitem em \lang:

\begin{itemize}
\item Variáveis ({\tt if a \{...\} }) e literais 
({\tt while 1 \{...\}});
\item Expressões binárias com os operadores: \\
{\tt + - * / \% > < >= <= == != \&\& ||};
\item Expressões unárias com os operadores:
	{\tt -} {\tt !};
\item Chamadas de função.
\end{itemize}

\pagebreak
A Listagem~\ref{lst:fat} mostra o código para o cálculo 
do fatorial como um exemplo de aplicação da linguagem 
\lang.

\begin{lstlisting}[
    label=lst:fat,
    caption={Exemplo de código usando a linguagem \lang{}.}
]
fn fatorial(n) {
        var
            i = 1;
            r = 1;
    
        if n < 0 {
            return -1;
        }

        while i < n + 1 {
            r = r * i;
            i++;
        }
        return r;
}
    
fn main() {
        var 
            i = 3;
            f = 0;
    
        f = fat(i);

        return 0;
}
\end{lstlisting}

\section*{Projeto}

A partir da definição da linguagem \lang:

\begin{enumerate}
\item Escreva uma analisador léxico usando o {\tt lex} 
para a linguagem;
\item Escreva um analisador sintático usando o {\tt yacc};
\item A partir do analisador sintático, imprima a tabela de 
símbolos para o código de entrada com os símbolos apresentados 
de acordo com o contexto. Por exemplo, para a Listagem~\ref{lst:fat}
 o símbolo {\tt i} da função {\tt fatorial} deve ser impresso como 
 {\tt fatorial.i}.
\end{enumerate}

O código da Listagem~\ref{lst:fat} pode ser usado como referência
nos testes. Se o arquivo contendo as regras léxicas chama-se 
{\tt xyz.l}, com as regras gramáticais {\tt xyz.y} e o 
código da Listagem~\ref{lst:fat} {\tt fat.xyz}, os seguintes 
commandos podem ser executados:

\begin{lstlisting}[language=bash]
lex -o xyz.yy.c xyz.l
yacc -d -o xyz.tab.c xyz.y
cc -o xyz xyz.tab.c
./xyz fat.xyz
\end{lstlisting}

\end{document}
